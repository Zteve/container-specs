\documentclass[a4paper,twoside,12pt]{article}
\usepackage{zed-cm}
\usepackage{graphicx}
\usepackage[nottoc,numbib]{tocbibind}
\markboth{Draft}{Version 0.1}
\pagestyle{myheadings}
\begin{document}
\parskip 2 pt
\parindent 10 pt

\def\Slash{\slash\hspace{0pt}}

\title{A Model of Linux Control Groups}

\author{
Glyn Normington\and
Steve Powell
}

\maketitle
% The following three commands ensure the title page is stamped as
% confidential without a page number. Page numbering is started at the
% table of contents.
\thispagestyle{myheadings}
\pagenumbering{roman}
\setcounter{page}{1}

%=============================================================================

This is a working document which serves to record the justifications, deliberations and results of an attempt to
specify precisely some (or all) of the features of \emph{Linux control groups} which can support robust and reliable development of \emph{Warden} and related products.

\paragraph{Current ``to-do'' list:}
\begin{itemize}
\item \emph{ops}--- describe more operations (e.g. on $CGroupsTasks$);
\item \emph{hierarchy}--- describe trees of $CGroup$s; (\textsc{Done})
\item \emph{subsystem}--- describe subsystems and attachment to $CGroup$s;
\item \emph{RedHat rule 2}--- make rule 2 into a constraint on subsystems+hierarchies;
\item \emph{RedHat rules}--- ensure all RedHat rules are represented correctly.
\end{itemize}


% Alt-Cmd-M -- \emph{}
% Alt-Cmd-Z -- \zed{}
% Alt-Cmd-X -- \axdef{}
% Alt-Cmd-S -- \schema{}
% Alt-Shift-Cmd-T -- \texttt{}

% Type checking hacks
\newcommand{\true}{true}
\newcommand{\false}{false}
\renewcommand{\emptyset}{\varnothing}
%=============================================================================

\clearpage
\tableofcontents

\clearpage
\pagenumbering{arabic}

%=============================================================================
\section{Introduction}

In order to implement some features and improvements to \emph{Warden} we intend to sharpen our understanding of Linux control groups and simultaneously document their use.

Linux control groups are part of the basis for the current implementation of Warden as well as of Linux Containers (LXC), Docker, and Google's
\textit{lmctfy} (``let me contain that for you'') project and so are important to the industry as well as to Cloud
Foundry.

In the process we intend to use our improved understanding to benefit:
\begin{itemize}
\item Linux Kernel documentation;
\item Warden \cite{warden} and Garden \cite{garden} development and exploitation; and
\item the wider development audience.
\end{itemize}
In particular, a clearer understanding of Warden/Garden and their basis in Linux control groups will put their development 
on a much firmer footing and enable key abstractions to be introduced which will provide a much-needed, stable structure upon which
functional enhancements and the likely future evolution of Linux control groups can rest.

These are the deliverables we plan:

\begin{itemize}
\item \emph{CGroupSpec}: a precise Z specification for control groups, and (at least) the \texttt{memory} subsystem, states;
\item an improved introduction (for Z0L readers\footnote{\emph{Z0L} refers to people who do not speak Z; Z is their \emph{zero}th language.}) of control groups, etc.;
\item \emph{WardenSpec}: a precise Z specification for a Warden API, using some or all of the \emph{CGroupSpec} state;
\item an english version of the same (for Z0L readers);
\item an implementation framework that maps \emph{WardenSpec} to \emph{CGroupSpec}; and
\item proposals for future development of Warden functionality by reference to these specs.
\end{itemize}
As input to this work we access the following sources of information:
\begin{description}
\item[Kernel specs] the english descriptions of control groups in the Kernel literature \cite{linuxgroups};
\item[Kernel code] the Linux Kernel \cite{linuxkernel};
\item[experiments] tests run against the CGroups kernel code (on a shared VM);
\item[RedHat article] an article about CGroups written by RedHat which provides some invariants of the pseudo-filesystem state \cite{rharticle}.
\end{description}

%----------------------------------------------------------------------------------------------------
\subsection{The Linux specification \cite{linuxgroups}}

In order to kick off the discussion (and to show what we're up against) we include the first section of \cite{linuxgroups} here:

{\small \begin{verbatim}
1.1 What are cgroups ?
----------------------

Control Groups provide a mechanism for aggregating/partitioning sets
of tasks, and all their future children, into hierarchical groups with
specialized behaviour.

Definitions:

A *cgroup* associates a set of tasks with a set of parameters for one
or more subsystems.

A *subsystem* is a module that makes use of the task grouping
facilities provided by cgroups to treat groups of tasks in
particular ways. A subsystem is typically a "resource controller" that
schedules a resource or applies per-cgroup limits, but it may be
anything that wants to act on a group of processes, e.g. a
virtualization subsystem.

A *hierarchy* is a set of cgroups arranged in a tree, such that
every task in the system is in exactly one of the cgroups in the
hierarchy, and a set of subsystems; each subsystem has system-specific
state attached to each cgroup in the hierarchy.  Each hierarchy has
an instance of the cgroup virtual filesystem associated with it.

At any one time there may be multiple active hierarchies of task
cgroups. Each hierarchy is a partition of all tasks in the system.

User-level code may create and destroy cgroups by name in an
instance of the cgroup virtual file system, specify and query to
which cgroup a task is assigned, and list the task PIDs assigned to
a cgroup. Those creations and assignments only affect the hierarchy
associated with that instance of the cgroup file system.

On their own, the only use for cgroups is for simple job
tracking. The intention is that other subsystems hook into the
generic cgroup support to provide new attributes for cgroups, such
as accounting/limiting the resources which processes in a cgroup can
access. For example, cpusets (see Documentation/cgroups/cpusets.txt)
allow you to associate a set of CPUs and a set of memory nodes with
the tasks in each cgroup.
\end{verbatim}}
Although this reads reasonably well (and by comparison to its peers it is a pretty good document) it is
full of specification sins\footnote{We don't mean this as an insult: without documents like these we couldn't do our work; and it is no worse---and in many ways better---than many such.}. For example, it is ambiguous concerning the number of and relationship between hierarchies; it
doesn't explain what the initial state of a control group is; it talks about tasks and processes as though they were the same without comment (we think they are the same, but we're not sure); the rules about the relationship between tasks and control groups sound contradictory; and so on.

We very much hope to address these puzzles (and more) in the following sections.

%=============================================================================
\section{Notes}

We gather here some observations, experiments and discussions which have informed our bit of research. This section may not survive the life of this specification, but we find it useful to try and document what we learn \emph{as we learn it}.

%----------------------------------------------------------------------------------------------------
\subsection{Mailing lists discussions}

It does not seem possible to define a \texttt{cgroup} hierarchy with no subsystems attached to it.
See the discussion \cite{noop}.

%----------------------------------------------------------------------------------------------------
\subsection{Experiments}
We performed some simple experiments on a (virtual, and hence infinitely refreshable) Linux machine.

% - - - - - - - - - - - - - - - - - - - - - - - - - - - - -
\subsubsection*{Multiple subsystems}

Attaching multiple subsystems to a single hierarchy:
{\small \begin{verbatim}
# cd /tmp/warden/cgroup
# mkdir test
# mount -t cgroup -o 'cpuset,blkio' none /tmp/warden/cgroup/test
# cd test
# mkdir parent
# cd parent
# echo 0 > cpuset.mems 
# echo 0 > cpuset.cpus
# echo $$ > tasks
# cat tasks
1840
2014
# cat /proc/1840/cgroup
13:blkio,cpuset:/parent
4:memory:/parent/child
3:devices:/
2:cpuacct:/
1:cpu:/
# cat /proc/cgroups
#subsys_name	hierarchy	num_cgroups	enabled
cpuset	13	2	1
cpu	1	1	1
cpuacct	2	1	1
memory	4	3	1
devices	3	1	1
freezer	0	1	1
blkio	13	2	1
perf_event	0	1	1
# cat /proc/mounts 
...
none /tmp/warden/cgroup tmpfs rw,relatime 0 0
none /tmp/warden/cgroup/cpu cgroup rw,relatime,cpu 0 0
none /tmp/warden/cgroup/cpuacct cgroup rw,relatime,cpuacct 0 0
none /tmp/warden/cgroup/devices cgroup rw,relatime,devices 0 0
none /tmp/warden/cgroup/memory cgroup rw,relatime,memory 0 0
none /tmp/warden/cgroup/test cgroup rw,relatime,blkio,cpuset 0 0
\end{verbatim}}

% - - - - - - - - - - - - - - - - - - - - - - - - - - - - -
\subsubsection*{Multiple hierarchies}

Attaching a single subsystem to multiple hierarchies:
{\small \begin{verbatim}
$ pwd   
/home/vagrant
$ mkdir mem1
$ mkdir mem2
$ sudo su
# mount -t cgroup -o memory none /home/vagrant/mem1
# mount -t cgroup -o memory none /home/vagrant/mem2
# cd mem1
# mkdir inst1  
# ls inst1 
cgroup.clone_children  memory.failcnt ...
# ls ../mem2
cgroup.clone_children  inst1 memory.limit_in_bytes ...
# cd inst1
# echo 1000000 > memory.limit_in_bytes 
# cat memory.limit_in_bytes 
1003520
# cat ../../mem2/inst1/memory.limit_in_bytes 
1003520
# echo $$ > tasks
# cat tasks
1365
1409
# cat ../../mem2/inst1/tasks
1365
1411
\end{verbatim}}

% - - - - - - - - - - - - - - - - - - - - - - - - - - - - -
\subsubsection*{Overlapping hierarchies}

Attaching a new subsystem to one of the hierarchies attached to an existing subsystem (this follows on from the previous experiment):
{\small \begin{verbatim}
# mount -t cgroup -o cpuset none /home/vagrant/mem1
# ls /home/vagrant/mem1
cpuset.cpus ... // No memory.* files!
# ls /home/vagrant/mem2
cgroup.clone_children  inst1 memory.limit_in_bytes ...
\end{verbatim}}

%=============================================================================
\section{Control groups}

We begin by attempting to describe control groups.

%----------------------------------------------------------------------------------------------------
\subsection{Fundamentals}
There are fundamental sets of items we will not look into, for example $TASK$s.
\begin{zed}
[TASK]
\end{zed}

% - - - - - - - - - - - - - - - - - - - - - - - - - - - - -
\subsubsection*{A single control group}

A single control group in isolation is just a (finite) collection of tasks, and we can document this:

\begin{schema}{CGroup}
taskset : \finset TASK
\end{schema}
We do not feel the need to elaborate on this (at the moment).

As a convenience, we define the (global) function $tasks$ which extracts the $taskset$ from a $CGroup$, and the
simple control group $EmptyCGroup$ which has no tasks in it:
\begin{zed}
tasks == (\lambda CGroup @ taskset)
\also
EmptyCGroup == (\mu CGroup | taskset = \emptyset )
\end{zed}

% - - - - - - - - - - - - - - - - - - - - - - - - - - - - -
\subsubsection*{Control group names}

Control groups have names, but these have a structure---a file directory-like structure---so we define them as \emph{sequences} of simple names (think of these as directory names). The latter we will not define further:

\begin{zed}
[SIMPLENAME]
\end{zed}
and control group names are sequences of these, for which we use the shorthand $CGPath$:

\begin{zed}
CGPath == \seq SIMPLENAME
\end{zed}

There is a special name for the first (initial) control group, the root of the structure. We define a constant for it:

\begin{axdef}{}
ROOTCGROUPNAME : CGPath
\where
ROOTCGROUPNAME = \langle \rangle
\end{axdef}

There may be many $CGroup$s in the system, and a collection of $CGroup$ with names must `fill out' the tree-structure with $CGPath$ names. We explain this with a $NamedCGroups$ collection:

\begin{schema}{NamedCGroups}
cg : CGPath \ffun CGroup
\where
ROOTCGROUPNAME \in \dom cg \\
\forall s : \dom cg @ \forall t : CGPath | t \subset s @ t \in \dom cg
\end{schema}
The constraints imply that there is  at least one $CGroup$ in the collection (the root) and that the names are `prefix-closed'. The latter constraint is equivalent to saying the names fit together in a tree-structure (as in a file-system). 
There is a $CGroup$ for every node in the tree, `named' by the sequence of simple names from the root to the node.

We have said nothing about the \emph{tasks} in the $CGroup$s in the named collection.
This comes next.

%----------------------------------------------------------------------------------------------------
\subsection{System tasks}
In order to describe the relationship between named control groups and tasks, we need a minimal model of the tasks of the system. We first postulate a (fixed) initial task:

\begin{axdef}{}
INITIALTASK : TASK
\end{axdef}
and then we say that the tasks of the system always include this initial task:

\begin{schema}{Tasks}
systasks : \finset TASK
\where
INITIALTASK \in systasks
\end{schema}
which has the (pleasing) consequence that there is always at least one task in the system.

% - - - - - - - - - - - - - - - - - - - - - - - - - - - - -
\subsubsection*{Initialisation}

When the system tasks are first created, then, this $INITIALTASK$ must exist, which is to say that there is at least one task. We simplify it by saying there is exactly one (though we are not usually around to see this natal state).

\begin{schema}{InitTasks}
Tasks'
\where
\# systasks' = 1
\end{schema}
We may not need to use this property directly.

% - - - - - - - - - - - - - - - - - - - - - - - - - - - - -
\subsubsection*{Creating a new task}

We do need to speak about new task creation, and so here is how we model it in the system:

\begin{schema}{CreateTask}
task? : TASK \\
parent? : TASK
\also
\Delta Tasks
\where
task? \notin systasks \\
parent? \in systasks \\
systasks' = systasks \cup \{ task? \}
\end{schema}
The new task is not already in the system; it has to have a (designated) parent, which is just any existing system task;
the new task is inserted into the system, without disturbing any other tasks.

% - - - - - - - - - - - - - - - - - - - - - - - - - - - - -
\subsubsection*{Destroying a task}

Task destruction is also needed in the context of control groups:

\begin{schema}{DestroyTask}
task? : TASK
\also
\Delta Tasks
\where
task? \in systasks \\
task? \neq INITIALTASK \\
systasks'  = systasks \setminus \{ task? \}
\end{schema}
The task being destroyed is already in the system; it must not be the system's initial task (this could have been a derived restriction, since $Tasks'$ already insists that $INITIALTASK$ must remain in $systasks'$);
the task is removed from the set of system tasks, without disturbing the other tasks.

Of course we know there is a parent/child relationship between tasks which the system knows and maintains. We don't think we need to know about that (except at create time) so we don't model 
it\footnote{If we need this it will become clear later when we cannot express a constraint, or cannot describe a change of state correctly.}.

%----------------------------------------------------------------------------------------------------
\subsection{Control group hierarchy}
Now we can combine the named control groups and system tasks and say how they should be related. At the same time we can describe a function $cgroup$ which maps each task to a control group in the collection.

We are defining a \emph{hierarchy} of control groups.

\begin{schema}{CGHierarchy}
Tasks \\
NamedCGroups
\also
cgroup : TASK \ffun CGPath
\where
tasks \circ cg \partition systasks
\also
\dom cgroup = systasks \\
\forall t : systasks @ t \in tasks ( cg (cgroup ~ t))
\end{schema}
Each task of the system is in precisely one control group, or, as expressed here, the control group tasks \emph{partition} the system tasks; \emph{and} there is a function ($cgroup$) which maps each system task to the (full) name ($CGPath$) of the control group in which it resides.

It is important to note that the existence of the function $cgroup$ is guaranteed by the partition requirement that accompanies it. It is what is called `derived state', and these assertions impose no extra constraints. In fact, the existence of $cgroup$ is \emph{equivalent} to the partition constraint.

It is legitimate to ask how a $CGHierarchy$ is initialised. We do so next.


%----------------------------------------------------------------------------------------------------
\subsection{Hierarchy operations}

It is natural to consider state changes that might `happen' to hierarchies. We will try to understand how tasks are moved between control groups in a hierarchy, how a new task is inserted (into the system, and therefore into the groups), how a new control group is created, and how tasks and control groups are deleted. Describing each of these operations precisely challenges our state descriptions and constraints---which is a good thing.

% - - - - - - - - - - - - - - - - - - - - - - - - - - - - -
\subsubsection*{Initialisation}

We first consider initialisation of a hierarchy. This is a sort of degenerate operation, as is common in Z specifications.

\begin{schema}{InitCGHierarchy}
CGHierarchy' \\
\Xi Tasks
\where
\exists CGroup' @ \\
( ~~~ cg' = \{ ~ ROOTCGROUPNAME \mapsto \theta CGroup' ~\} \\
\land taskset' = systasks' ~~)
\end{schema}
We here \emph{decide} that all existing system tasks are placed in the initial $CGroup$, 
with the standard root name\footnote{Although we have decided this, 
we do not yet know if this is true. However, an assertion like this has the enormous advantage that it can be verified 
and tested, as can all of the state descriptions. If necessary we can come back and change it.}. 
Good job we declared it beforehand!

% - - - - - - - - - - - - - - - - - - - - - - - - - - - - -
\subsubsection*{Create a new $CGroup$ in a hierarchy}

How does a new control group arise (in a hierarchy)?
Clearly we need to name it, and it must have an initial state, but there are also things to do with the
hierarchy that have to be true, and we mustn't disturb the task structure of the system when
creating a new control group, either:

\begin{schema}{CreateCGroup}
path? : CGPath
\also
\Delta CGHierarchy \\
\Xi Tasks
\where
path? \notin \dom cg \\
path? \neq ROOTCGROUPNAME \\
cg' = cg \cup \{ path? \mapsto EmptyCGroup \} \\
cgroup' = cgroup
\end{schema}
The control group path name is not an existing path (it's not in the old state); it cannot be the root; the new path must point to the empty control group, and the path must `fit' into the $cg$ map (see below); the task partition is unaffected.

What does it mean for the path to `fit'? It means that all its prefixes are in $\dom cg$ already. Equivalently, its immediate parent (the `front' of the sequence) is there.

The constraint which says `$path?$ is not the root', is strictly redundant (it can be derived from the rest of the constraints).

% - - - - - - - - - - - - - - - - - - - - - - - - - - - - -
\subsubsection*{Delete a $CGroup$ from a hierarchy}

When a control group is deleted, what happens to its tasks or its children (in the hierarchy)?
This is explain here by saying that you can only destroy control groups that don't have tasks or children.
Here is how this can be formalised:

\begin{schema}{DestroyCGroup}
path? : CGPath
\also
\Delta CGHierarchy \\
\Xi Tasks
\where
path? \notin \dom cg' \\
path? \neq ROOTCGROUPNAME \\
cg = cg' \cup \{ path? \mapsto EmptyCGroup \} \\
cgroup' = cgroup
\end{schema}
The constraints say: this is not a new path (it's not in the new state); it cannot be the root; the path identifies the empty control group (in the old state); the task partition is unaffected.

Notice that the path \emph{used to}`fit' in $cg$, and if removing it leaves the $cg'$ structured correctly, this means it cannot have had any children.

In this case the constraint which says $path?$ is not the root, is \emph{not} redundant.

The constraints for $Create$... and $Destroy$... are deliberately symmetric.
Doing it this way actually helped us to (write and) understand them.

% \begin{schema}{NewCGroupTask}
% \end{schema}
% \begin{schema}{DeleteCGroupTask}
% \end{schema}

% - - - - - - - - - - - - - - - - - - - - - - - - - - - - -
\subsubsection*{Move a task from one $CGroup$ to another}

Within a hierarchy we want to explain what happens if we move a task from one control group to another.

The system tasks are not changed by this, and we need to preserve the partitioning nature of the hierarchy. We need to supply the task which we are moving ($t?$), and the (full name of the) control group we are moving it to ($dest?$). We do not need to supply the control group it `moves from', because the hierarchy state has that information already.

\begin{schema}{MoveCGroupTask}
t? : TASK \\
dest? : CGPath
\also
\Delta CGHierarchy \\
\Xi Tasks 
\where
t? \in systasks \\
dest? \in \dom cg
\also
\dom cg = \dom cg' \\
cgroup' = cgroup \oplus \{ ~t? \mapsto dest? ~\}
\end{schema}
The constraints on the inputs are that the task exists already and that the name is the name of a known control group (of this hierarchy). The new map $cgroup'$ is the same as the old one, except that the input task, $t?$, is now mapped to the control group named $dest?$, and no more control groups are introduced by this operation.

Amazingly, this is all we need to say.

The map $cgroup'$, which we noted above was \emph{derived state}, uniquely determines the partition (except for
empty control groups) 
and \emph{vice versa}, so simply saying what the new map should be (and that no extra empty control groups arise) suffices to determine the rest of the new state, with the partition constraint holding.

%=============================================================================
\section{Subsystems}

tbd

\clearpage

\appendix

\begin{flushleft}
\begin{thebibliography}{99}   % `99' is a picture of the maximum generated numeric references -- they are two digits in this bibliography

%%  Example bibliography entry:
%\bibitem{knuth76} % citation callout, e.g.: \cite{knuth76}
%  Donald E. Knuth, % author
%  \emph{The computer as Master Mind}. % title
%  J. Recreational Mathematics, Vol.~9(1), 1976-1977. % publisher, or journal, volume and date

\bibitem{warden}
  Various authors,
  \emph{Warden github repository},
  \texttt{https://github.com/cloudfoundry/warden}.

\bibitem{garden}
  Various authors,
  \emph{Garden github repository}, 
  \texttt{https://github.com/pivotal-cf-experimental/garden}.

% API abstracting the cgroup filesystem
% See http://manpages.ubuntu.com/manpages/lucid/man5/cgconfig.conf.5.html for configuration containing useful insights
\bibitem{libcgroup}
  Various authors
  \emph{libcgroup},
  {\small \texttt{http://libcg.sourceforge.net/html/index.html}}.

\bibitem{linuxgroups}
  Paul Menage, Paul Jackson and Christoph Lameter,
  \emph{CGROUPS},
  {\small \texttt{https://www.kernel.org/doc/Documentation/cgroups/cgroups.txt}}, 2004-2006.

\bibitem{linuxkernel}
  Linus Torvalds, \emph{et al},
  \emph{Linux kernel source tree},
  {\small \texttt{https://github.com/torvalds/linux}}.

\bibitem{memory}
  Various authors,
  \emph{Memory Resource Controller},
  {\small \texttt{https://www.kernel.org/doc/Documentation/cgroups/memory.txt}}.

\bibitem{noop}
  Kamezawa Hiroyuki, \emph{et al},
  \emph{NOOP cgroup subsystem},
  {\small \texttt{http://thread.gmane.org/gmane.linux.kernel/777763}}.

\bibitem{rharticle}
  Martin Prpi\u{c}, R\"udiger Landmann, and Douglas Silas,
  \emph{Red Hat Enterprise Linux 6.5 GA: Resource Management Guide}, 
  {\small \texttt{https://access.redhat.com/site/documentation/en-US\slash{}Red\_Hat\_Enterprise\_Linux/6/pdf/Resource\_Management\_Guide\slash{}Red\_Hat\_Enterprise\_Linux-6-Resource\_Management\_Guide-en-US.pdf}}.

\end{thebibliography}
\end{flushleft}
\end{document}
