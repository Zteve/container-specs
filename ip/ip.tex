\documentclass[a4paper,twoside,12pt]{article}
\usepackage{../z/zed-cm}
\usepackage{graphicx}
\usepackage[nottoc,numbib]{tocbibind}
\markboth{Draft}{Version 0.1}
\pagestyle{myheadings}
\begin{document}
\parskip 2 pt
\parindent 10 pt

\def\Slash{\slash\hspace{0pt}}

\title{Container Networking}

\author{
Glyn Normington\and
Steve Powell
}

\maketitle
% The following three commands ensure the title page is without a page number but page numbering starts here.
% Page numbers appear on subsequent pages, and are roman until the main body, which starts again at arabic 1.
\thispagestyle{empty}
\pagenumbering{roman}
\setcounter{page}{1}

%=============================================================================

This document describes the container networking spec.

% Alt-Cmd-M -- \emph{}
% Alt-Cmd-Z -- \zed{}
% Alt-Cmd-X -- \axdef{}
% Alt-Cmd-S -- \schema{}
% Alt-Shift-Cmd-T -- \texttt{}

% Type checking hacks
\newcommand{\true}{true}
\newcommand{\false}{false}
\renewcommand{\emptyset}{\varnothing}
%=============================================================================

\clearpage
\tableofcontents

\cleardoublepage
\pagenumbering{arabic}
\setcounter{page}{1}

%=============================================================================
\section{Introduction}

This is a document that records the authors' deliberations as they come to grips with ``what networking really
does''. It arises out of a need to control the networking of containers, including creating several containers which share a subnet, and the restriction of access of containers to the wider internet.

%=============================================================================
\section{Overview of this document}

This document is a place to store our initial thoughts as we investigate what the internet (IP) stack is like. The intention is to find the right decomposition of ideas to simply describe the state, and state transitions, of the components in the stack, and the tasks that they perform.

The document currently focuses on container networking and does not yet model the underlying IP and link layers. Behaviour in error cases is not specified; error cases are modelled simply as false preconditions.

The Z specification language is used to capture the model, but sufficient English text is also provided that readers who do not know Z should be able to understand the model. The appendix contains a summary of the Z notation.

%=============================================================================
\section{Fundamentals}

Nodes of the IP network are identified by IP address. Containers are identified by a container id. A subnet is a set of IP addresses. The formats of IP addresses and the structure of subnets need not be modelled for our current purposes.
\begin{zed}
[ IpAddr, ContainerId ] 
\also
Subnet == \power IpAddr
\end{zed}

%=============================================================================
\newpage
\section{Network pool}

A network pool is a managed collection of IP addresses which enables IP addresses to be allocated and freed. The pool is partitioned into allocated and free subsets.
\begin{schema}{NetworkPool}
pool : \power IpAddr \\
alloc, free : \power IpAddr
\where
free \subseteq pool \\
alloc = pool \setminus free
\end{schema}

%=============================================================================
\section{Network pool operations}

Operations on a network pool do not modify the pool of IP addresses.
\begin{schema}{NetworkPoolChange}
\Delta NetworkPool
\where
pool' = pool
\end{schema}

A single IP address can be allocated from the network pool.
\begin{schema}{Allocate}
NetworkPoolChange \\
ip! : IpAddr
\where
ip! \in free \\
free' = free \setminus \{ ip! \}
\end{schema}

An IP address and a gateway address can be allocated from the network pool.
\begin{schema}{AllocateTwo}
NetworkPoolChange \\
ip!, gw! : IpAddr
\where
ip! \in free \land gw! \in free \\
ip! \neq gw! \\
free' = free \setminus \{ ip!, gw! \}
\end{schema}
The allocated addresses are distinct.
\\ \\
A set of IP addresses can be returned to the network pool. Any which did not come from the
network pool are ignored.
\begin{schema}{Release}
NetworkPoolChange \\
ips? : \power IpAddr
\where
alloc' = alloc \setminus ips?
\end{schema}

%=============================================================================
\newpage
\section{Container Network}

A container network has an IP address, it belongs to a subnet, and it has a gateway address which it uses to route packets of data, known as \emph{datagrams},  to non-local IP addresses.
The container IP address identifies the container from a networking perspective as we will see later.
\begin{schema}{ContainerNet}
ip : IpAddr \\
subnet : Subnet \\
gw: IpAddr
\where
ip \in subnet \\
ip \neq gw \\
gw \notin subnet
\end{schema}
The container's IP address belongs to the container's subnet. The gateway address is distinct from the container's IP address and does not belong to the container's subnet.

\subsection{Invariants}

Each $ContainerNet$ has a relationship with a particular $NetworkPool$. In practice, a network pool is associated with zero or more container networks.

$CNetPooled$ holds when the container $ip$ is not part of a special subnet, and is allocated from the pool.
\begin{schema}{CNetPooled}
ContainerNet \\
NetworkPool
\where
ip \in pool \\
subnet = \{ ip \} \\
gw \in alloc
\end{schema}
The container's gateway is allocated from the pool.

$CNetSubnet$ holds when the container $ip$ is part of a subnet potentially shared with other containers.
\begin{schema}{CNetSubnet}
ContainerNet \\
NetworkPool
\where
subnet \cap pool = \emptyset \\
gw \in alloc
\end{schema}
The container's gateway is allocated from the pool, but the container's IP address is not.

A container network is valid with respect to a network pool if either the container $ip$ is allocated from the pool or is part of a special subnet.
\begin{zed}
CNetValid \defs CNetPooled \lor CNetSubnet
\end{zed}

There are also invariants between distinct pairs of $ContainerNet$s.
\begin{zed}
CNetPair \defs ContainerNet_1 \land ContainerNet_2 \\
CNetDistinctPair \defs [ CNetPair | ip_1 \neq ip_2 ]
\end{zed}
Any distinct pair of container networks have distinct container IP addresses since the container IP address identifies the container from a networking perspective.

Some distinct pairs of container networks share a special subnet.
\begin{schema}{CNetPairShared}
CNetDistinctPair
\where
subnet_1 = subnet_2 \\
gw_1 \neq gw_2
\end{schema}
Such pairs of container networks have distinct gateway addresses.

Other distinct pairs of container networks have disjoint subnets. It turns out that at most one of the pair has a special subnet and at least one of the pair has a container IP address allocated from the network pool.
\begin{schema}{CNetPairDisjoint}
CNetDistinctPair
\where
subnet_1 \cap subnet_2 = \emptyset \\
gw_1 \neq gw_2
\end{schema}
These pairs of container networks also have distinct gateway addresses.

A distinct pair of container networks is valid providing the pair shares a special subnet or they have disjoint subnets. 
\begin{zed}
CNetPairValid \defs CNetPairShared \lor CNetPairDisjoint
\end{zed}

%=============================================================================
\newpage
\section{Network}

The network, at least from the point of view of containers, has a network pool and a collection of container networks indexed by container id.
\begin{schema}{Net}
NetworkPool \\
cnet : ContainerId \pfun ContainerNet
\where
\forall cn : \ran cnet; ContainerNet | cn = \theta ContainerNet \\
	\t1 @ CNetValid \\
\also
\forall c_1, c_2 : \dom cnet ; CNetPair \\
	\t1 | c_1 \neq c_2 \land cnet~c_1 = \theta ContainerNet_1 \land cnet~c_2 = \theta ContainerNet_2 \\
	\t1 @ CNetPairValid
\end{schema}
All the container networks in the collection are valid with respect to the network pool and with respect to each other.

%=============================================================================
\section{Network operations}

The following sections model operations on the network.
\\ \\
All operations on the network preserve the network pool of IP addresses (although IP addresses may be allocated and freed).
\begin{zed}
NetChange \defs \Delta Net \land NetworkPoolChange
\end{zed}

\subsection{Creating a container}

We now specify what it means to create a container, from a networking perspective.
\\ \\
The basic principle of creating a container is that a new container id is generated and associated with a container network. Notice that the new container network is valid with respect to the network pool and with respect to any previously existing container networks.
\begin{schema}{CNCreateBase}
NetChange \\
ContainerNet
\also
cid! : ContainerId
\where
cid! \notin \dom cnet \\
cnet' = cnet \cup \{ cid! \mapsto \theta ContainerNet \}
\end{schema}

There are three ways of creating a container network: the container IP address may be allocated automatically from the network pool; a specific container IP address may be allocated from the network pool; or a container network may be created for a special subnet. These variants are distinguished syntactically by the input parameters and are modelled below as three separate operations.

Firstly, a container network may be created by allocating its container IP address from the network pool.
\begin{schema}{CNCreateFromPool}
CNCreateBase
\where
AllocateTwo [ ip / ip!, gw / gw! ] \\
subnet = \{ ip \}
\end{schema}
The container's IP and gateway addresses are allocated from the network pool. The container's subnet consists of the container's IP address alone.

Secondly, a container network may be created by allocating a container IP address specified as an input parameter from the network pool.
\begin{schema}{CNCreateFromPoolWithSpecificIP}
CNCreateFromPool \\
ip? : IpAddr
\where
ip? = ip
\end{schema}
The specified IP address must be free for the operation to succeed.

Thirdly, a container network may be allocated with a special subnet and a container IP address from that subnet. This variant enables multiple containers to be created which share a subnet.
\begin{schema}{CNCreateSubnet}
CNCreateBase \\
subnet? : Subnet \\
ip? : IpAddr
\where
ip? \in subnet? \\
ip? = ip \\
subnet? = subnet \\
Allocate[ gw / ip! ]
\end{schema}
The gateway address is allocated from the network pool. The invariants in $Net$ imply that the specified subnet is either new or is identical to that of one or more existing containers in which case those containers have container IP addresses distinct from the specified container IP address. In other words, it is an error to specify a subnet and a container IP address that are already in use.

\subsection{Destroying a container}
Destroying a container from a network perspective simply means removing it from the collection of containers and releasing the IP addresses that were allocated for the container from the network pool.
\begin{schema}{CNDestroy}
NetChange \\
cid? : ContainerId
\where
\exists ContainerNet | cid? \mapsto \theta ContainerNet \in cnet @ \\
	\t1 cnet' = \{ cid? \} \ndres cnet \land \\
	\t1 (\exists ips : \power IpAddr | ips = \{ ip, gw \} @ \\
		\t2 Release[ ips / ips? ])
\end{schema}

%%%%%%%%%%%%%%%%%%%%%%%%%%%%%%%%%%%%%%%%%%%%%%%
%   A P P E N D I C E S
%%%%%%%%%%%%%%%%%%%%%%%%%%%%%%%%%%%%%%%%%%%%%%%

\clearpage

\appendix

%=============================================================================
%   Z   N O T A T I O N
%=============================================================================
\section{Z Notation}
\label{sec:znot}
{\scriptsize
\makeatletter % the following code is taken from Mike Spivey's zed.tex

\def\symtab{\setbox0=\vbox\bgroup \def\\{\cr}
        \halign\bgroup\strut$##$\hfil&\quad##\hfil\cr}
\def\endsymtab{\crcr\egroup\egroup
        \dimen0=\ht0 \divide\dimen0 by2 \advance\dimen0 by\ht\strutbox
        \splittopskip=\ht\strutbox \vbadness=10000
        \predisplaypenalty=0
        $$\halign{##\cr\hbox to\linewidth{%
                \valign{##\vfil\cr
                        \setbox1=\vsplit0 to\dimen0 \unvbox1\cr
                        \noalign{\hfil}\unvbox0\cr
                        \noalign{\hfil}}}\cr
                \noalign{\prevdepth=\dp\strutbox}}$$
        \global\@ignoretrue}

\makeatother

Numbers:
\begin{symtab}
        \nat & \verb/Natural numbers/ \{\verb/0,1,.../\} \\
%       \num & \verb/Integers (...,-1,0,1,...)/ \\
%       \nat_1 & \verb/Positive natural numbers/ \\
%       \upto & \verb/integral range/ \\
%       + & \verb/Addition/\quad\hfill 3 \\
%       - & \verb/Subtraction/\quad\hfill 3 \\
%       * & \verb/Multiply/\quad\hfill 4 \\
%       \div & \verb/Remainder/\quad\hfill 4 \\
%       \mod & \verb/Modulus/\quad\hfill 4 \\
%       < & \verb/Less than/ \\
%       > & \verb/Greater than/ \\
%       \leq & \verb/Less than or equal/ \\
%       \geq & \verb/Greater than or equal/ \\
%       \neq & \verb/Inequality/ \\
\end{symtab}
Propositional logic and the schema calculus:
\begin{symtab}
%       \lnot & \verb/Not/ \\
        \ldots\land\ldots & \verb/And/ \\
        \ldots\lor\ldots & \verb/Or/ \\
        \ldots\implies\ldots & \verb/Implies/ \\
%       \iff & \verb/If and only if/ \\
        \forall..\mid..\spot.. & \verb/For all/ \\
        \exists..\mid..\spot.. & \verb/There exists/ \\
%       \exists_1..\mid..\spot.. & \verb/There exists unique/ \\
        \ldots\hide\ldots & \verb/Hiding/ \\
%       \project & \verb/\project/ \\
%       \pre & \verb/\pre/ \\
%       \semi & \verb/\semi/
        \ldots\defs\ldots & \verb/Schema definition/ \\
        \ldots==\ldots & \verb/Abbreviation/ \\
        \ldots::=\ldots\mid\ldots & \verb/Free type definition/ \\
        \ldata\ldots\rdata & \verb/Free type injection/ \\
        [\ldots] & \verb/Given sets/ \\
        ',?,!,_0\ldots_9 & \verb/Schema decorations/ \\
        \ldots\shows\ldots & \verb/theorem/ \\
        \theta\ldots & \verb/Binding formation/ \\
        \lambda\ldots & \verb/Function definition/ \\
        \mu\ldots & \verb/Mu-expression/ \\
        \Delta\ldots & \verb/State change/ \\
        \Xi\ldots & \verb/Invariant state change/ \\
\end{symtab}
Sets and sequences:
%and bags:
\begin{symtab}
        \{\ldots\} & \verb/Set/ \\
        \{..\mid..\spot..\} & \verb/Set comprehension/ \\
        \power\ldots & \verb/Set of subsets of/ \\
%       \power_1 & \verb/Non-empty subsets of/ \\
%       \finset & \verb/Finite sets/ \\
%       \finset_1 & \verb/Non-empty finite sets/ \\
        \emptyset & \verb/Empty set/ \\
        \ldots\cross\ldots & \verb/Cartesian product/ \\
        \ldots\in\ldots & \verb/Set membership/ \\
        \ldots\notin\ldots & \verb/Set non-membership/ \\
        \ldots\cup\ldots & \verb/Union/ \\
        \ldots\cap\ldots & \verb/Intersection/ \\
        \ldots\setminus\ldots & \verb/Set difference/ \\
        \bigcup\ldots & \verb/Distributed union/ \\
%       \bigcap & \verb/Distributed intersection/ \\
        \#\ldots & \verb/Cardinality/ \\
%       \dcat & \verb/Distributed sequence concatenation/
        \ldots\subseteq\ldots & \verb/Subset/ \\
        \ldots\subset\ldots & \verb/Proper subset/ \\
        \ldots\partition\ldots & \verb/Set partition/ \\
        \seq & \verb/Sequences/ \\
%       \seq_1 & \verb/Non-empty sequences/ \\
%       \iseq & \verb/Injective sequences/ \\
        \langle\ldots\rangle & \verb/Sequence/ \\
%       \cat & \verb/Sequence concatenation/ \\
        \disjoint\ldots & \verb/Disjoint sequence of sets/ \\
%       \bag & \verb/Bags/ \\
%       \lbag\ldots\rbag & \verb/Bag/ \\
%       \inbag & \verb/Bag membership/ \\
\end{symtab}
%Here are the infix function symbols. Each symbol is
%shown with its priority:
%\begin{symtab}
%       \uplus & \verb/\uplus/ \\
%       \filter & \verb/Schema projection/ \\
%       \uminus & \verb/\uminus/
%\end{symtab}
Functions and relations:
\begin{symtab}
        \ldots\rel\ldots\quad\quad~~~  & \verb/Relation/ \\
        \ldots\pfun\ldots & \verb/Partial function/ \\
        \ldots\fun\ldots  & \verb/Total function/ \\
        \ldots\pinj\ldots & \verb/Partial injection/ \\
        \ldots\inj\ldots  & \verb/Injection/ \\
%       \psurj & \verb/Partial surjection/ \\
%       \surj & \verb/Surjection/ \\
%       \bij  & \verb/Bijection/ \\
%       \ffun & \verb/Finite partial function/ \\
%       \finj & \verb/Finite partial injection/ \\
        \dom\ldots & \verb/Domain/ \\
        \ran\ldots & \verb/Range/ \\
        \ldots\mapsto\ldots & \verb/maplet/ \\
        \ldots\inv & \verb/Relational inverse/ \\
%       \ldots\plus & \verb/Transitive closure/ \\
        \ldots\star & \verb/Reflexive-transitive/ \\
        ~           & \verb/closure/ \\
%       \ldots\bsup n \esup & \verb/Relational iteration/ \\
        \ldots\limg\ldots\rimg & \verb/Relational image/ \\
%       \comp & \verb/Forward relational composition/ \\
%       \circ & \verb/Relational composition/ \\
        \ldots\oplus\ldots & \verb/Functional overriding/ \\
        \ldots\dres\ldots & \verb/Domain restriction/ \\
        \ldots\rres\ldots & \verb/Range restriction/ \\
        \ldots\ndres\ldots & \verb/Domain subtraction/ \\
        \ldots\nrres\ldots & \verb/Range subtraction/ \\
%       \id & \verb/Identity relation/ \\
\end{symtab}
Axiomatic descriptions:
%%unchecked
\begin{axdef}
  Declarations
\where
  Predicates
\end{axdef}
Schema definitions:
%%unchecked
\begin{schema}{SchemaName}
  Declaration
\where
  Predicates
\end{schema}
}
\newpage

%=============================================================================
%   B I B L I O G R A P H Y
%=============================================================================
%\newpage
%\begin{flushleft}
%\begin{thebibliography}{99}
%\label{sec:references}
% `99' is a picture of the generated numeric references -- they are two digits in this bibliography
% If we had a hundred or more we would have used 999, or whatever.

%%  Example bibliography entry:
%\bibitem{knuth76}                                                        % citation callout, e.g.: \cite{knuth76}
%  Donald E. Knuth,                                                        % author
%  \emph{The computer as Master Mind}.                    % title
%  J. Recreational Mathematics, Vol.~9(1), 1976-1977. % publisher, or journal, volume and date

%\end{thebibliography}
%\end{flushleft}
\end{document}
