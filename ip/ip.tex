\documentclass[a4paper,twoside,12pt]{article}
\usepackage{../z/zed-cm}
\usepackage{graphicx}
\usepackage[nottoc,numbib]{tocbibind}
\markboth{Draft}{Version 0.1}
\pagestyle{myheadings}
\begin{document}
\parskip 2 pt
\parindent 10 pt

\def\Slash{\slash\hspace{0pt}}

\title{Container networking}

\author{
Glyn Normington\and
Steve Powell
}

\maketitle
% The following three commands ensure the title page is without a page number but page numbering starts here.
% Page numbers appear on subsequent pages, and are roman until the main body, which starts again at arabic 1.
\thispagestyle{empty}
\pagenumbering{roman}
\setcounter{page}{1}

%=============================================================================

This document describes the container networking spec.

% Alt-Cmd-M -- \emph{}
% Alt-Cmd-Z -- \zed{}
% Alt-Cmd-X -- \axdef{}
% Alt-Cmd-S -- \schema{}
% Alt-Shift-Cmd-T -- \texttt{}

% Type checking hacks
\newcommand{\true}{true}
\newcommand{\false}{false}
\renewcommand{\emptyset}{\varnothing}
%=============================================================================

\clearpage
\tableofcontents

\cleardoublepage
\pagenumbering{arabic}
\setcounter{page}{1}

%=============================================================================
\section{Introduction}

This is a document that records the deliberations of Glyn and Steve as they come to grips with ``what networking really
does''. It arises out of a need to control the networking of containers, including creating several containers which share a subnet, and the restriction of access of containers to the wider internet.

%=============================================================================
\section{Overview of this document}

This document is a place to store our initial thoughts as we investigate what the internet (ip) stack is like. The intention is to find the right decomposition of ideas to simply describe the state, and state transitions, of the components in the stack, and the tasks that they perform.

%=============================================================================
\section{Fundamentals}

\begin{zed}
[ IpAddr, ContainerId ] 
\also
Subnet == \power IpAddr
\end{zed}

\section{Network states}

Explain

\begin{schema}{ContainerNet}
ip : IpAddr \\
subnet : Subnet \\
gw: IpAddr
\where
ip \in subnet \\
ip \neq gw \\
gw \notin subnet
\end{schema}

\begin{schema}{NetworkPool}
pool : \power IpAddr \\
alloc, free : \power IpAddr
\where
free \subseteq pool \\
alloc = pool \setminus free
\end{schema}

\section{Network pool operations}

\begin{schema}{NetworkPoolChange}
\Delta NetworkPool
\where
pool' = pool
\end{schema}

\begin{schema}{Allocate}
NetworkPoolChange \\
ip! : IpAddr
\where
ip! \in free \\
free' = free \setminus \{ ip! \}
\end{schema}

\begin{schema}{AllocateTwo}
NetworkPoolChange \\
ip!, gw! : IpAddr
\where
ip! \in free \land gw! \in free \\
ip! \neq gw! \\
free' = free \setminus \{ ip!, gw! \}
\end{schema}

\subsection{Invariants}

Each $ContainerNet$ has a relationship with the $NetworkPool$. $CNetPooled$ holds when the container $ip$ is not part of a special subnet, and is allocated from the pool.

$CNetSubnet$ holds when the container $ip$ is part of a subnet potentially shared with other containers.

\begin{schema}{CNetSubnet}
ContainerNet \\
NetworkPool
\where
subnet \cap pool = \emptyset \\
gw \in alloc
\end{schema}

\begin{schema}{CNetPooled}
ContainerNet \\
NetworkPool
\where
ip \in pool \\
subnet = \{ ip \} \\
gw \in alloc
\end{schema}

\begin{zed}
CNetValid \defs CNetPooled \lor CNetSubnet
\end{zed}

We also have invariants that hold between pairs of $ContainerNet$s.

\begin{zed}
CNetPair \defs ContainerNet_1 \land ContainerNet_2 \\
CNetDistinctPair \defs [ CNetPair | ip_1 \neq ip_2 ]
\end{zed}

\begin{schema}{CNetPairShared}
CNetDistinctPair
\where
subnet_1 = subnet_2 \\
gw_1 \neq gw_2
\end{schema}

\begin{schema}{CNetPairDisjoint}
CNetDistinctPair
\where
subnet_1 \cap subnet_2 = \emptyset \\
gw_1 \neq gw_2
\end{schema}

\begin{zed}
CNetPairValid \defs CNetPairShared \lor CNetPairDisjoint
\end{zed}

\section{More network state}

\begin{schema}{Net}
NetworkPool \\
cnet : ContainerId \pfun ContainerNet
\where
\forall cn : \ran cnet; ContainerNet | cn = \theta ContainerNet \\
	\t1 @ CNetValid \\
\forall c_1, c_2 : \dom cnet ; CNetPair \\
	\t1 | c_1 \neq c_2 \land cnet~c_1 = \theta ContainerNet_1 \land cnet~c_2 = \theta ContainerNet_2 \\
	\t1 @ CNetPairValid
\end{schema}

\begin{zed}
NetChange \defs \Delta Net \land NetworkPoolChange
\end{zed}

\section{Creating a container}

\begin{schema}{CNCreateBase}
NetChange \\
ContainerNet
\also
cid! : ContainerId
\where
cid! \notin \dom cnet \\
cnet' = cnet \cup \{ cid! \mapsto \theta ContainerNet \}
\end{schema}

\begin{schema}{CNCreateFromPool}
CNCreateBase
\where
AllocateTwo [ ip / ip!, gw / gw! ] \\
subnet = \{ ip \}
\end{schema}

%%%%%%%%%%%%%%%%%%%%%%%%%%%%%%%%%%%%%%%%%%%%%%%
%   A P P E N D I C E S
%%%%%%%%%%%%%%%%%%%%%%%%%%%%%%%%%%%%%%%%%%%%%%%

\clearpage

\appendix

%=============================================================================
%   Z   N O T A T I O N
%=============================================================================
\section{Z Notation}
\label{sec:znot}
{\scriptsize
\makeatletter % the following code is taken from Mike Spivey's zed.tex

\def\symtab{\setbox0=\vbox\bgroup \def\\{\cr}
        \halign\bgroup\strut$##$\hfil&\quad##\hfil\cr}
\def\endsymtab{\crcr\egroup\egroup
        \dimen0=\ht0 \divide\dimen0 by2 \advance\dimen0 by\ht\strutbox
        \splittopskip=\ht\strutbox \vbadness=10000
        \predisplaypenalty=0
        $$\halign{##\cr\hbox to\linewidth{%
                \valign{##\vfil\cr
                        \setbox1=\vsplit0 to\dimen0 \unvbox1\cr
                        \noalign{\hfil}\unvbox0\cr
                        \noalign{\hfil}}}\cr
                \noalign{\prevdepth=\dp\strutbox}}$$
        \global\@ignoretrue}

\makeatother

Numbers:
\begin{symtab}
        \nat & \verb/Natural numbers/ \{\verb/0,1,.../\} \\
%       \num & \verb/Integers (...,-1,0,1,...)/ \\
%       \nat_1 & \verb/Positive natural numbers/ \\
%       \upto & \verb/integral range/ \\
%       + & \verb/Addition/\quad\hfill 3 \\
%       - & \verb/Subtraction/\quad\hfill 3 \\
%       * & \verb/Multiply/\quad\hfill 4 \\
%       \div & \verb/Remainder/\quad\hfill 4 \\
%       \mod & \verb/Modulus/\quad\hfill 4 \\
%       < & \verb/Less than/ \\
%       > & \verb/Greater than/ \\
%       \leq & \verb/Less than or equal/ \\
%       \geq & \verb/Greater than or equal/ \\
%       \neq & \verb/Inequality/ \\
\end{symtab}
Propositional logic and the schema calculus:
\begin{symtab}
%       \lnot & \verb/Not/ \\
        \ldots\land\ldots & \verb/And/ \\
        \ldots\lor\ldots & \verb/Or/ \\
        \ldots\implies\ldots & \verb/Implies/ \\
%       \iff & \verb/If and only if/ \\
        \forall..\mid..\spot.. & \verb/For all/ \\
        \exists..\mid..\spot.. & \verb/There exists/ \\
%       \exists_1..\mid..\spot.. & \verb/There exists unique/ \\
        \ldots\hide\ldots & \verb/Hiding/ \\
%       \project & \verb/\project/ \\
%       \pre & \verb/\pre/ \\
%       \semi & \verb/\semi/
        \ldots\defs\ldots & \verb/Schema definition/ \\
        \ldots==\ldots & \verb/Abbreviation/ \\
        \ldots::=\ldots\mid\ldots & \verb/Free type definition/ \\
        \ldata\ldots\rdata & \verb/Free type injection/ \\
        [\ldots] & \verb/Given sets/ \\
        ',?,!,_0\ldots_9 & \verb/Schema decorations/ \\
        \ldots\shows\ldots & \verb/theorem/ \\
        \theta\ldots & \verb/Binding formation/ \\
        \lambda\ldots & \verb/Function definition/ \\
        \mu\ldots & \verb/Mu-expression/ \\
        \Delta\ldots & \verb/State change/ \\
        \Xi\ldots & \verb/Invariant state change/ \\
\end{symtab}
Sets and sequences:
%and bags:
\begin{symtab}
        \{\ldots\} & \verb/Set/ \\
        \{..\mid..\spot..\} & \verb/Set comprehension/ \\
        \power\ldots & \verb/Set of subsets of/ \\
%       \power_1 & \verb/Non-empty subsets of/ \\
%       \finset & \verb/Finite sets/ \\
%       \finset_1 & \verb/Non-empty finite sets/ \\
        \emptyset & \verb/Empty set/ \\
        \ldots\cross\ldots & \verb/Cartesian product/ \\
        \ldots\in\ldots & \verb/Set membership/ \\
        \ldots\notin\ldots & \verb/Set non-membership/ \\
        \ldots\cup\ldots & \verb/Union/ \\
        \ldots\cap\ldots & \verb/Intersection/ \\
        \ldots\setminus\ldots & \verb/Set difference/ \\
        \bigcup\ldots & \verb/Distributed union/ \\
%       \bigcap & \verb/Distributed intersection/ \\
        \#\ldots & \verb/Cardinality/ \\
%       \dcat & \verb/Distributed sequence concatenation/
        \ldots\subseteq\ldots & \verb/Subset/ \\
        \ldots\subset\ldots & \verb/Proper subset/ \\
        \ldots\partition\ldots & \verb/Set partition/ \\
        \seq & \verb/Sequences/ \\
%       \seq_1 & \verb/Non-empty sequences/ \\
%       \iseq & \verb/Injective sequences/ \\
        \langle\ldots\rangle & \verb/Sequence/ \\
%       \cat & \verb/Sequence concatenation/ \\
        \disjoint\ldots & \verb/Disjoint sequence of sets/ \\
%       \bag & \verb/Bags/ \\
%       \lbag\ldots\rbag & \verb/Bag/ \\
%       \inbag & \verb/Bag membership/ \\
\end{symtab}
%Here are the infix function symbols. Each symbol is
%shown with its priority:
%\begin{symtab}
%       \uplus & \verb/\uplus/ \\
%       \filter & \verb/Schema projection/ \\
%       \uminus & \verb/\uminus/
%\end{symtab}
Functions and relations:
\begin{symtab}
        \ldots\rel\ldots\quad\quad~~~  & \verb/Relation/ \\
        \ldots\pfun\ldots & \verb/Partial function/ \\
        \ldots\fun\ldots  & \verb/Total function/ \\
        \ldots\pinj\ldots & \verb/Partial injection/ \\
        \ldots\inj\ldots  & \verb/Injection/ \\
%       \psurj & \verb/Partial surjection/ \\
%       \surj & \verb/Surjection/ \\
%       \bij  & \verb/Bijection/ \\
%       \ffun & \verb/Finite partial function/ \\
%       \finj & \verb/Finite partial injection/ \\
        \dom\ldots & \verb/Domain/ \\
        \ran\ldots & \verb/Range/ \\
        \ldots\mapsto\ldots & \verb/maplet/ \\
        \ldots\inv & \verb/Relational inverse/ \\
%       \ldots\plus & \verb/Transitive closure/ \\
        \ldots\star & \verb/Reflexive-transitive/ \\
        ~           & \verb/closure/ \\
%       \ldots\bsup n \esup & \verb/Relational iteration/ \\
        \ldots\limg\ldots\rimg & \verb/Relational image/ \\
%       \comp & \verb/Forward relational composition/ \\
%       \circ & \verb/Relational composition/ \\
        \ldots\oplus\ldots & \verb/Functional overriding/ \\
        \ldots\dres\ldots & \verb/Domain restriction/ \\
        \ldots\rres\ldots & \verb/Range restriction/ \\
        \ldots\ndres\ldots & \verb/Domain subtraction/ \\
        \ldots\nrres\ldots & \verb/Range subtraction/ \\
%       \id & \verb/Identity relation/ \\
\end{symtab}
Axiomatic descriptions:
%%unchecked
\begin{axdef}
  Declarations
\where
  Predicates
\end{axdef}
Schema definitions:
%%unchecked
\begin{schema}{SchemaName}
  Declaration
\where
  Predicates
\end{schema}
}
\newpage

%=============================================================================
%   B I B L I O G R A P H Y
%=============================================================================
\newpage
\begin{flushleft}
\begin{thebibliography}{99}
\label{sec:references}
% `99' is a picture of the generated numeric references -- they are two digits in this bibliography
% If we had a hundred or more we would have used 999, or whatever.

%%  Example bibliography entry:
%\bibitem{knuth76}                                                        % citation callout, e.g.: \cite{knuth76}
%  Donald E. Knuth,                                                        % author
%  \emph{The computer as Master Mind}.                    % title
%  J. Recreational Mathematics, Vol.~9(1), 1976-1977. % publisher, or journal, volume and date

\end{thebibliography}
\end{flushleft}
\end{document}
